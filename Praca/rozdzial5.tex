\chapter{Podsumowanie oraz kierunki dalszych prac}
W ramach niniejszej pracy magisterskiej zaimplementowano aplikację służącą do rozpoznawania gestów dłoni działającą w czasie rzeczywistym. Oprogramowanie miało na celu rozpoznawanie kilku prostych gestów otrzymanych za pomocą urządzenia do rejestracji strumienia wideo. W pierwszym etapie pracy dokonano analiza istniejących rozwiązań dostępnych na rynku komercyjnym. Okazało się, że nie znaleziono aplikacji, która spełniałaby cele postawione w niniejszej pracy. Kolejny etap to przegląd istniejących metod do rozpoznawania oraz klasyfikacji gestów. Dzięki takiej analizie dokonano wyboru metody służącej do znajdowania fragmentów obrazu o kolorystyce ludzkiej skóry. Zdecydowano się na ekstrakcję takich regionów za pomocą filtracji obrazu w przestrzeni barw HSL. Model ten jest zdecydowanie bardziej odporny na zmiany oświetlenia niż klasyczna przestrzeń RGB. Następnie zaproponowano wykorzystanie algorytmu SURF do detekcji oraz opisu punktów kluczowych obrazu. Metoda ta została wykorzystana w wielu projektach starających się rozwiązać problem opisu cech obrazu za pomocą wektora cech. W niniejszej pracy wykorzystano model SVM jako klasyfikator dla próbek testowych. Sukces klasyfikacji w ogromnej mierze zależał od parametrów  wybranych dla modelu SVM.

Działanie zaimplementowanej metody rozpoznawania gestów zaprezentowano w rozdziale \ref{cha: Tests}. 

Badano wpływ zmian poprawności działania metody ze względu na różne wartości parametrów, do których należały: długość wektora opisującego obraz, rodzaj metody jądra modelu SVM, jego tolerancję oraz wartość współczynnika funkcji kary. Testy przeprowadzono również ze względu na różne zbiory próbek uczących. Dla bazy danych, w której zdjęcia obejmowały jedynie dłoń, najlepsze rezultaty otrzymano dla ..., gdzie poprawność klasyfikacji wynosiła ...  W przypadku zbioru gestów, dla których 

Testowanie zaproponowanej metody rozpoznawania gestów przedstawiono w rozdziale \ref{cha: Tests}. 

