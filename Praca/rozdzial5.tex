\chapter{Podsumowanie oraz kierunki dalszych prac}
W ramach niniejszej pracy magisterskiej zaimplementowano aplikację służącą do rozpoznawania gestów dłoni działającą w czasie rzeczywistym. Oprogramowanie miało na celu rozpoznawanie kilku prostych gestów otrzymanych za pomocą urządzenia do rejestracji strumienia wideo. W pierwszym etapie pracy dokonano analizy istniejących rozwiązań dostępnych na rynku komercyjnym. Okazało się, że nie znaleziono aplikacji, która spełniałaby cele postawione w niniejszej pracy. Kolejny etap to przegląd istniejących metod do rozpoznawania oraz klasyfikacji gestów. Dzięki takiej analizie dokonano wyboru metody służącej do znajdowania fragmentów obrazu o kolorystyce ludzkiej skóry. Zdecydowano się na ekstrakcję takich regionów za pomocą filtracji obrazu w przestrzeni barw HSL. Model ten jest zdecydowanie bardziej odporny na zmiany oświetlenia niż klasyczna przestrzeń RGB. Następnie zaproponowano wykorzystanie algorytmu SURF do detekcji oraz opisu punktów kluczowych obrazu. Metoda ta została wykorzystana w wielu projektach starających się rozwiązać problem opisu cech obrazu za pomocą wektora cech. W niniejszej pracy wykorzystano model SVM jako klasyfikator dla próbek testowych. Sukces klasyfikacji w ogromnej mierze zależał od parametrów  wybranych dla modelu SVM.

W rozdziale \ref{cha: Tests} badano poprawność działania metody ze względu na różne wartości parametrów, do których należały: długość wektora opisującego obraz, rodzaj metody jądra modelu SVM, jego tolerancję oraz wartość współczynnika funkcji kary. Dla wybranej bazy danych najlepsze rezultaty otrzymano dla parametrów: 
\begin{itemize}
	\item Liczba klastrów: 50
	\item Wartość współczynnika funkcji kary: 100
	\item Tolerancja: 0.00001
	\item Funkcja jądra przekształcenia dla klasyfikatora: RBF
	\item Parametr $\sigma$: 20	
\end{itemize}
Dla tak przyjętych parametrów poprawność klasyfikacji dla wszystkich gestów wynosiła: 97.2\% dla obrazów treningowych oraz 94\% dla próbek testowych.

\newpage
Do najważniejszych elementów dalszych prac nad aplikacją do rozpoznawania gestów dłoni należą:
\begin{enumerate}
	\item Rozbudowanie istniejącej bazy danych o nowe gesty.
	\item Przeniesienie bazy zawierającej gesty na chmurę.
	\item Dodanie możliwości wyboru innej metody do ekstrakcji cech obrazu.
	\item Implementacja dodatkowych funkcji \textit{kernel trick'u}.
	\item Dodanie możliwości wyboru innych klasyfikatorów.
	\item Wykorzystanie bazy danych, gdzie dłoń nie jest jedyną częścią ciała człowieka (np. obraz dłoni wraz z głową) oraz zaproponowanie algorytmu do eliminacji fałszywych obiektów.
	\item Dodanie funkcjonalności do usuwania pojedynczego obrazu lub całego gestu z bazy danych.
\end{enumerate} 

