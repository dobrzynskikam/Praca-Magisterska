\chapter{Testy oraz prezentacja wyników}
\label{cha: Tests}
Testy sprawdzające działanie aplikacji zostały przeprowadzone za pomocą komputera wyposażonego w czterordzeniowy procesor Intel\textregistered CORE\texttrademark i7 oraz pamięć RAM 16GB DDR3. W ramach testów wykorzystano jedną z ogólnie dostępnych baz danych zawierające gesty dłoni w zmieniających się warunkach oświetlenia.

W ramach testów wykorzystano bazę gestów opsianą w pracy BIBLIOGRAFIA. Zawiera ona 900 sekwencji obrazów tworząc 9 różnych klas. Klasy te są utworzone na podstawie 3 prostych gestów dłoni, która jest poddawana rotacji oraz zaciskaniu w pięść. Szczegóły na rysunku RYSUNEK. Każda z klas zawiera 100 sekwencji obrazów (5 odmiennych warunków oświetleniowych, 10 różnych sekwencji ruchu dłoni w dwóch kierunkach). Każdy z gestów jest umieszczony na ciemnym tle bez dodatkowych elementów.

Stworzono łącznie 33 sekwencje testowe. Różnią się one względem siebie wartościami poszczególnych parametrów. Pierwszy podział dotyczy ilości klastrów, na jakie podzielona zostanie przestrzeń zmiennych opisujących cechy obrazu. Wybrano 3 wartości: 10, 36 oraz 50. Dla każdej z nich przeprowadzono następujące sekwencje testowe: 
\begin{enumerate}
	\item Wartość współczynnika funkcji kary: 100; Funkcja jądra: wielomianowa; Stopień wielomianu: 1; Stała wielomianu: 1; 
	\begin{enumerate}
		\item Tolerancja: 0.00001
		\item Tolerancja: 0.01
		\item Tolerancja: 10
	\end{enumerate}
	\item Wartość współczynnika funkcji kary: 100; 
	Tolerancja: 0.00001; Funkcja jądra: wielomianowa;
	\begin{enumerate}
		\item Stopień wielomianu: 1; Stała wielomianu: 1;
		\item Stopień wielomianu: 1; Stała wielomianu: 50;
		\item Stopień wielomianu: 1; Stała wielomianu: 1000;
		\item Stopień wielomianu: 3; Stała wielomianu: 1;
		\item Stopień wielomianu: 10; Stała wielomianu: 1;
	\end{enumerate}
	\item Wartość współczynnika funkcji kary: 100; 
	Tolerancja: 0.00001; Funkcja jądra: metoda RBF;
	\begin{enumerate}
		\item Sigma: 10
		\item Sigma: 20
		\item Sigma: 5
	\end{enumerate}
\end{enumerate}

\begin{table}
	\centering
	\begin{tabular}{|l|c|c|c|c|}
		\hline
		 					& \textbf{Palm} & \textbf{Fist} & \textbf{Five Fingers} & \textbf{Two Fingers} \\ \hline
		Tolerancja=0.00001 	& 88.6\% 		& 94.7\%		& 87.2\%				& 98.3\% \\ \hline
		Tolerancja=0.01 	& 88.6\% 		& 94.7\%		& 87.2\%				& 98.3\% \\ \hline
		Tolerancja=10		& 100\%			& 0\%			& 0\%					& 0\% \\ \hline
		\multirow{4}{*}{Pozostałe parametry} & \multicolumn{4}{|l|}{Liczba klastrów: 36} \\ 
							& \multicolumn{4}{|l|}{Współczynnik funkcji kary: 100} \\ 
							& \multicolumn{4}{|l|}{Metoda funkcji jądra: wielomianowa} \\
							& \multicolumn{4}{|l|}{Stopień: 1; Stała: 1} \\ \hline
	\end{tabular}
	\caption{Proba.}
\end{table}


\begin{table}
	\centering
	\begin{tabular}{|l|c|c|c|c|}
		\hline
		& \textbf{Palm} & \textbf{Fist} & \textbf{Five Fingers} & \textbf{Two Fingers} \\ \hline
		Stopień=1; Stała=1 	& 88.6\% 		& 94.7\%		& 87.2\%				& 98.3\% \\ \hline
		Stopień=1; Stała=50 	& 88.6\% 		& 94.7\%		& 87.2\%				& 98.3\% \\ \hline
		Stopień=1; Stała=1000		& 100\%			& 0\%			& 0\%					& 0\% \\ \hline
		Stopień=3; Stała=1		& 100\%			& 0\%			& 0\%					& 0\% \\ \hline
		\multirow{3}{*}{Pozostałe parametry} & \multicolumn{4}{|l|}{Liczba klastrów: 36} \\ 
		& \multicolumn{4}{|l|}{Współczynnik funkcji kary: 100} \\ 
		& \multicolumn{4}{|l|}{Tolerancja: 0.00001} \\ \hline
	\end{tabular}
	\caption{Proba.}
\end{table}


\begin{table}
	\centering
	\begin{tabular}{|l|c|c|c|c|}
		\hline
		& \textbf{Palm} & \textbf{Fist} & \textbf{Five Fingers} & \textbf{Two Fingers} \\ \hline
		Sigma=10 	& 88.6\% 		& 94.7\%		& 87.2\%				& 98.3\% \\ \hline
		Sigma=20  	& 88.6\% 		& 94.7\%		& 87.2\%				& 98.3\% \\ \hline
		Sigma=5 		& 100\%			& 0\%			& 0\%					& 0\% \\ \hline
		\multirow{2}{*}{Pozostałe} & \multicolumn{4}{|l|}{Liczba klastrów: 36} \\ 
		\multirow{2}{*}{parametry} & \multicolumn{4}{|l|}{Współczynnik funkcji kary: 100} \\ 
		& \multicolumn{4}{|l|}{Tolerancja: 0.00001} \\ \hline
	\end{tabular}
	\caption{Proba.}
\end{table}

%Testy sprawdzające działanie aplikacji zostały przeprowadzone za pomocą komputera wyposażonego w czterordzeniowy procesor Intel\textregistered CORE\texttrademark i7 oraz pamięć RAM 16GB DDR3. W ramach testów wykorzystano ogólnie dostępne bazy danych zawierające gesty dłoni w zmieniających się warunkach oświetlenia. 
%Pierwszą z wykorzystanych baz była baza danych wykorzystana w pracy BIBLIOGRAFIA. Zawiera ona 900 sekwencji obrazów tworząc 9 różnych klas. Klasy te są utworzone na podstawie 3 prostych gestów dłoni, która jest poddawana rotacji oraz zaciskaniu w pięść. Szczegóły na rysunku RYSUNEK. Każda z klas zawiera 100 sekwencji obrazów (5 odmiennych warunków oświetleniowych, 10 różnych sekwencji ruchu dłoni w dwóch kierunkach). Każdy z gestów jest umieszczony na ciemnym tle bez dodatkowych elementów. 

%Na potrzeby pracy z bazy wybrano 5 różnych gestów, do których należały: dłoń ze (i) złączonymi oraz (ii) rozłączonymi palcami, (iii) pięść oraz dwa palce (iv) złączone oraz (v) rozłączone. 

%W ramach testów sprawdzono poprawność klasyfikacji gestu dla zmieniających się wartosci parametrów. Parametry poddane testom to: długość wektora opisującego gest, wybór metody \textit{kernel trick} (wraz z wpływem wartości poszczególnych parametrów), zmiany parametru funkcji kary oraz tolerancji.

%Dla każdego gestu 70\% wszystkich próbek zostało użyte w celu uczenia klasyfikatora, reszta to część testowa. 

%Dla każdego testu sprawdzono poprawność algorytmu rozpoznawania gestów dla część treningowej oraz testowej.  


%BOW = 36 Wielomianowa deg 1 con 1 compl= 1 tol = 1 czas wyliczenia poprawnosc testowa i treningowa

%BOW = 12 Wielomianowa deg 1 con 1 compl= 1 tol = 1 czas wyliczenie

%BOW = 72 Wielomianowa deg 1 con 1 compl= 1 tol = 1 czas wyliczenie

%BOW = 36 Gauss sigma = 1 compl= 1 tol = 1 czas wyliczenia

%BOW = 12 Gauss sigma = 1 compl= 1 tol = 1 czas wyliczenie

%BOW = 72 Gauss sigma = 1 compl= 1 tol = 1 czas wyliczenie

%Kolejna baza danych to zbiór obrazów dostępny 

%Ostatni etap to testowanie aplikacji wykorzystując urządzenie do rejestracji strumienia wideo. Do testów wykorzystano kamerę KAMERA operującą w rozdzielczości ROZDZIELCZOSC. 