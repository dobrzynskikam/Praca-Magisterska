\chapter{Wprowadzenie}
\label{cha:wprowadzenie}
Rozpoznawanie gestów dłoni to jedno z najbardziej obiecujących metod służących do interakcji pomiędzy człowiekiem a maszyną. Gest ten można zdefiniować jako ciągły w czasie zbiór sekwencji ludzkiej dłoni. Rozpoznawanie gestów jest szczególnie wykorzystywane w komunikacji pomiędzy ludźmi mającymi problemy ze słuchem. Dodatkowo znajduje zastosowanie w przypadku obserwacji oraz monitoringu wideo jak również w zdalnym sterowaniu urządzeń (roboty, telewizory, konsole do gier). 

W początkowej fazie rozwoju tej techniki, rozpoznawanie gestów było realizowane przy pomocy specjalnych rękawic lub znaczników umieszczonych na opuszkach palców. Jednakże z powodu, że użytkowanie takiego interfejsu jest niewygodne i nienaturalne, większość obecnych prac koncentruje się na podejściu opartym o systemy wideo rejestrujące gołą dłoń.

Wydajny interfejs człowiek-maszyna (z ang. \textit{Human Computer Interface}, HCI) jest w stanie rozpoznać gest ręki w czasie rzeczywistym. Jednakże śledzenie oraz rozpoznawanie gestów oparte o system wideo musi sprostać wielu wymagającym problemom, których źródłem jest złożoność gestów wynikająca z dużej ilości punktów swobody ludzkiej dłoni. Dodatkowo system czasu rzeczywistego musi poprawnie odseparować dłoń od reszty tła, spełniać wymagania odnośnie czasu obliczeń oraz poprawności klasyfikacji.

Ludzka dłoń może zostać odseparowana od innych elementów obrazu na podstawie informacji o kolorze skóry. Ta technika jest szeroko stosowana w kontekście detekcji dłoni (\cite{Temp1} oraz \cite{Temp2}). Twórcy pracy \cite{SiftBowSvm} zaproponowali sposób rozpoznawania gestów dłoni działający w czasie rzeczywistym. Pierwszy etap projektu zakładał wykrycie koloru skóry poprzez transformację obrazu gestu z przestrzeni barw RGB do modelu HSL (z ang. \textit{Hue, Saturation, Luminance}). Przestrzeń ta jest bardziej odporna na zmianę natężenia światła dla obserwowanej dłoni. Dodatkowo zastosowali metodę wykrywania twarzy opisaną w pracy \cite{ViolaJonesRobustDetection}, aby można było ją usunąć w momencie wyliczania wektora cech dla dłoni. Pozostający obraz został poddany działaniu algorytmu SIFT (z ang. \textit{Scale Invariance Feature Transform}), który wyliczał wektor cech opisujący każdy obraz. Ostatni etap to proces tworzenia modelu SVM służący do klasyfikacji próbek testowych. Projekt utworzony w niniejszej pracy został w dużym stopniu oparty o zasadę działania opisaną w bieżącym akapicie.

Na rynku istnieje kilka firm zajmujących się tworzeniem oprogramowania pomagającego osobom głuchym bądź słabo słyszącym. Jedną z najciekawszych aplikacji do rozpoznawania języka migowego dostępnej na rynku jest produkt o nazwie \textit{UNI} autorstwa firmy \textit{Motionsavvy} z siedzibą w Rochester, Stany Zjednoczone . Jest to firma zajmująca się rozwiązywaniem problemów z komunikacją dla ludzi mających problemy ze słuchem. Od 2011 roku \textit{Motionsavvy} rozbudowuje bazę danych dla gestów Amerykańskiego Języka Migowego (z ang. \textit{American Sign Language}, ASL). Rozwijana przez nich aplikacja umożliwia komunikację osób niesłyszących poprzez rozpoznawanie gestów obu rąk oraz palców. Dodatkowo daje możliwość tłumaczenia mowy na język migowy. Aplikacja zawiera funkcjonalności umożliwiające nagrywanie, etykietowanie oraz edycję gestów wprowadzonych przez użytkownika. Zawiera również słownik, w którym każdy użytkownik aplikacji ma możliwość dodawania własnych gestów. Oprogramowanie zostało przetestowane oraz zainstalowane na lotnisku w Rochester, które co roku obsługuje największy odsetek ludzi słabo słyszących w całych Stanach Zjednoczonych. Problematyczny jest fakt, ze na chwilę obecną (Czerwiec 2018) aplikacja nie jest dostępna dla użytkowników komercyjnych. Niemożliwe zatem jest przetestowanie aplikacji pod względem poprawności klasyfikacji nowych gestów.
 
%Aplikacja jest dostępna do pobrania na urządzenia mobilne z systemem Android bądz iOS. Widok aplikacji przedstawiono na rysunku RYSUNEK. Umożliwia tłumaczenie tekstu bądź mowy na Amerykański Język Migowy (z ang. \textit{American Sign Language}, ASL) wykorzystując animowanego awatara 3D pokazującego wynik tłumaczenia. Dodatkowo zawiera słownik z ponad 3000 znaków dla ASL wspomagający naukę języka migowego za pomocą telefonu. Nie daje ona jednak możliwości 

%Wykrycie koloru skóry następuje poprzez transformację przestrzeni barw RGB do modelu HSL (z ang. \textit{Hue, Saturation, Luminance}). Przestrzeń ta jest bardziej odporna na zmianę natężenia światła dla obserwowanej dłoni. 

\section{Cele pracy}
\label{sec:celePracy}
Celem pracy było stworzenie aplikacji do rozpoznawania gestów dłoni działające w czasie rzeczywistym. Dodatkowo należało opracować mechanizm do definiowania własnych gestów, edytowanie oraz usuwania istniejących. W tym celu należało zapoznać się z bibliotekami dla języka C\# zawierającymi implementację potrzebnych metod pomocnych podczas tworzenia aplikacji. Następnie aplikacja została poddana testom, których celem było sprawdzenie poprawności implementacji oraz jej szybkości działania. Testy wykonano ze względu na metodę wyboru algorytmu oraz przy zmieniających się wartościach parametrów.
%---------------------------------------------------------------------------

\section{Zawartość pracy}
\label{sec:zawartoscPracy}
Niniejsza praca składa się z 5 rozdziałów:
\begin{enumerate}
	\item Wstęp - wprowadzenie do rozpoznawania gestów, ukazanie istoty oraz podkreślenie możliwości, jakie może przynieść zaimplementowana aplikacja
	\item Pojęcia związane z analizowanym problemem, przedstawienie jednej z najbardziej popularnych metod służących do opisu cech obrazu. Dodatkowo w rozdziale znajduje się opis wykorzystanych technologii oraz narzędzi, które okazały się pomocne w trakcie implementacji.
	\item W rozdziale 3 przedstawiono architekturę aplikacji wraz z dokładnym opisem poszczególnych składowych programu.
	\item Testy porównujące działanie aplikacji dla rożnych wartości parametrów.
	\item Podsumowanie: ostatni rozdział zawiera podsumowanie całej pracy oraz wskazuje kierunki dalszego rozwoju aplikacji. 
\end{enumerate}
%W rodziale~\ref{cha:pierwszyDokument} przedstawiono podstawowe informacje dotyczące struktury dokumentów w \LaTeX u. Alvis~\cite{Alvis2011} jest językiem 


















