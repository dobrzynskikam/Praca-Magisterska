\chapter{Wprowadzenie}
\label{cha:wprowadzenie}
Rozpoznawanie gestów dłoni to jedna z najbardziej obiecujących metod służących do interakcji pomiędzy człowiekiem a maszyną. Gest ten można zdefiniować jako ciągły w czasie zbiór sekwencji ludzkiej dłoni. Rozpoznawanie gestów jest szczególnie wykorzystywane w komunikacji pomiędzy ludźmi mającymi problemy ze słuchem. Dodatkowo znajduje zastosowanie w przypadku obserwacji oraz monitoringu wideo jak również w zdalnym sterowaniu urządzeń (roboty, telewizory, konsole do gier). 

W początkowej fazie rozwoju tej techniki rozpoznawanie gestów było realizowane przy pomocy specjalnych rękawic lub znaczników umieszczonych na opuszkach palców. Jednakże z powodu, że użytkowanie takiego interfejsu jest niewygodne i nienaturalne, większość obecnych prac koncentruje się na~podejściu opartym o systemy wideo rejestrujące gołą dłoń.

Wydajny interfejs człowiek-maszyna (z ang. \textit{Human Computer Interface}, w skrócie HCI) to taki system, który jest w stanie rozpoznać gest ręki działając w czasie rzeczywistym. Jednakże śledzenie oraz rozpoznawanie gestów oparte o system wideo musi sprostać wielu wymagającym problemom, których źródłem jest złożoność gestów wynikająca z dużej ilości punktów swobody ludzkiej dłoni. Dodatkowo system czasu rzeczywistego musi poprawnie odseparować dłoń od reszty tła, spełniać wymagania odnośnie czasu obliczeń oraz poprawności klasyfikacji.

Ludzka dłoń może zostać odseparowana od innych elementów obrazu na podstawie informacji o~kolorze skóry. Ta technika jest szeroko stosowana w kontekście detekcji dłoni (\cite{Temp1} oraz \cite{Temp2}). Twórcy pracy \cite{SiftBowSvm} zaproponowali sposób rozpoznawania gestów dłoni działający w czasie rzeczywistym. Pierwszy etap projektu zakładał wykrycie koloru skóry poprzez transformację obrazu gestu z przestrzeni barw RGB do~modelu HSL (skrót od ang. \textit{Hue, Saturation, Luminance}). Przestrzeń ta jest bardziej odporna na zmianę natężenia światła dla obserwowanej dłoni. Dodatkowo zastosowali metodę wykrywania twarzy opisaną w pracy \cite{ViolaJonesRobustDetection}, aby można było ją usunąć w momencie wyliczania wektora cech dla dłoni. Pozostający obraz został poddany działaniu algorytmu SIFT (skrót od ang. \textit{Scale Invariance Feature Transform}) \cite{Sift}, który wyliczał wektor cech opisujący pojedynczy obraz. Kolejno każdy rezultat działania algorytmu SIFT został poddany grupowaniu, aby wszystkie elementy były takiej samej długości. Ostatni etap to proces tworzenia modelu SVM służący do klasyfikacji próbek testowych. Algorytm rozpoznawania gestów dłoni zaproponowany w niniejszej pracy został w dużym stopniu oparty o zasadę działania opisaną w pracy \cite{SiftBowSvm}.

Na rynku istnieje kilka firm zajmujących się tworzeniem oprogramowania pomagającego osobom głuchym bądź słabo słyszącym. Jedną z najciekawszych aplikacji do rozpoznawania języka migowego dostępnej na rynku jest produkt o nazwie \textit{UNI} autorstwa firmy \textit{Motionsavvy} z siedzibą w Rochester, Stany Zjednoczone \cite{Uni}. Jest to organizacja zajmująca się rozwiązywaniem problemów z komunikacją dla ludzi mających problemy ze słuchem. Od 2011 roku \textit{Motionsavvy} rozbudowuje bazę danych dla gestów Amerykańskiego Języka Migowego (z ang. \textit{American Sign Language}, w skrócie ASL). Rozwijana przez nich aplikacja umożliwia komunikację osób niesłyszących poprzez rozpoznawanie gestów obu rąk oraz palców. Dodatkowo daje możliwość tłumaczenia mowy na język migowy. Aplikacja zawiera funkcjonalności umożliwiające nagrywanie, etykietowanie oraz edycję gestów wprowadzonych przez użytkownika. Zawiera również słownik, w którym każdy użytkownik aplikacji ma możliwość dodawania własnych gestów. Oprogramowanie zostało przetestowane oraz zainstalowane na lotnisku w Rochester, które co roku obsługuje największy odsetek ludzi słabo słyszących w całych Stanach Zjednoczonych. Problematyczny jest fakt, że na chwilę obecną (Czerwiec 2018) aplikacja nie jest dostępna dla użytkowników komercyjnych. Niemożliwe zatem jest przetestowanie aplikacji pod względem poprawności klasyfikacji gestów.

\section{Cele pracy}
\label{sec:celePracy}
Celem pracy było stworzenie aplikacji do rozpoznawania gestów dłoni działające w czasie rzeczywistym. W tym celu opracowano mechanizm do definiowania własnych gestów oraz dodawania nowych rekordów do już istniejących elementów. Kolejno należało zapoznać się z bibliotekami dla języka C\# zawierającymi implementację potrzebnych metod pomocnych podczas tworzenia aplikacji. Dodatkowo zaproponowano schemat architektury dla tworzonego oprogramowania jak również opracowano algorytm służący do rozpoznawania gestów oraz ich klasyfikacji. Następnie aplikacja została poddana testom, których celem było sprawdzenie poprawności zaproponowanej implementacji. Testy wykonano ze względu na ilość klastrów dla algorytmu k-średnich oraz dla zmieniających się wartości parametrów klasyfikatora.
%---------------------------------------------------------------------------

\section{Zawartość pracy}
\label{sec:zawartoscPracy}
Niniejsza praca składa się z 5 rozdziałów:
\begin{enumerate}
	\item Wstęp - wprowadzenie do rozpoznawania gestów, ukazanie istoty oraz podkreślenie możliwości, jakie może przynieść zaimplementowana aplikacja. Dodatkowo opisano przykład działania algorytmu służącego do rozpoznawania gestów dłoni działającego w czasie rzeczywistym. Oprócz tego wspomniano o przykładzie firmy, która na swoim koncie ma udany projekt aplikacji pomagającej osobom głuchym lub słabosłyszącym w komunikacji.
	\item Pojęcia związane z analizowanym problemem, przedstawienie jednej z najbardziej popularnych metod służących do opisu cech obrazu. Dodatkowo w rozdziale znajduje się opis wykorzystanych technologii oraz narzędzi, które okazały się pomocne w trakcie implementacji oprogramowania.
	\item W rozdziale 3 przedstawiono architekturę aplikacji wraz z dokładnym opisem poszczególnych składowych programu. Opisano najważniejsze wzorce projektowe wykorzystane w projekcie.
	\item Testy porównujące działanie aplikacji dla różnych wartości parametrów.
	\item Ostatni rozdział zawiera podsumowanie całej pracy oraz wskazuje kierunki dalszego rozwoju aplikacji. 
\end{enumerate}